\section{Bookkeeping}
\label{sec:bookkeeping}

\subsection{Naive Cartographer code}
\label{subsec:code}

The Naive Cartographer Python package is hosted
\href{
https://codeberg.org/brohrer/cartographer
}{in a Codeberg repository}
with mirrors
\href{
https://gitlab.com/brohrer/cartographer
}{on GitLab} and
\href{
https://github.com/brohrer/cartographer
}{GitHub}.

The README of the repo has instructions for installation and
examples of how to integrate it into your project.

\subsection{Versions}
\label{subsec:versions}

The \textbf{\href{
https://codeberg.org/brohrer/cartographer-paper/src/branch/main/cartographer.pdf}
{latest version}}
of this document and all the files needed to
render it are in \textbf{\href{
https://codeberg.org/brohrer/cartographer-paper}
{this Codeberg repository}}. There's a backup copy in \textbf{
\href{https://github.com/brohrer/cartographer-paper}
{this repository on GitHub}} and 
 \textbf{
\href{https://gitlab.com/brohrer/cartographer-paper}
{this repository on GitLab}}.

I don't expect this doc to ever be done. I'm always learning new things,
or thinking of a better way to explain something, or I do a new
piece of work I can't help myself from including. And there's always
one more bug.
Since it's a git repository, you are free to browse past commits to watch
the evolution, but I'll try to keep a running record of important updates
here.

% \begin{itemize}
% \item{\textbf{\href{
% https://codeberg.org/brohrer/ziptie-paper/src/commit/758b34e0749b75f35c98818d11d41553a8828b48/ziptie.pdf}
% {December 2, 2023}}. Rough outline of how Ziptie works
% and how it's related to the rest of the algorithmic world.}
% \end{itemize}

% \subsection{History}
% \label{subsec:history}

% \begin{itemize}

% \item{\textbf{2011}. As the method was taking shape, I published a 
% flurry of posters and write-ups in small conferences: 
% \end{itemize}

\subsection{Citations}
\label{subsec:citations}

If you end up using Naive Cartographer in your work, give it a shout out.
Here's an APA example you can copy and paste. (You may have to fiddle with
the dates.)

Rohrer, B. (2024). Naive Cartographer: A Reinforcement Learner [White Paper].
Retrieved \today, from
\href{https://brandonrohrer.com/cartographer}{https://brandonrohrer.com/cartographer}


\subsection{Licensing}
\label{subsec:license}

The text, figures, equations, and methods described in this paper
are published under the CC0 ``No Rights Reserved" license.
From the
\href{https://creativecommons.org/public-domain/cc0/}{Creative Commons} description,
CC0 ``enables scientists, educators,
artists and other creators and owners of copyright- or database-protected
content to waive those interests in their works and thereby place them as
completely as possible in the public domain, so that others may freely
build upon, enhance and reuse the works for any purposes without restriction
under copyright or database law."

\begin{figure}[ht]
\vskip 0.2in
\begin{center}
\centerline{\includegraphics[width=1.0in]{images/cc-zero.png}}
\label{fig:cc0}
\end{center}
\vskip -0.2in
\end{figure}

\subsection{Contact Me}
\label{subsec:contact}

I'm at brohrer@gmail.com. You're welcome to email me at any time for
any reason. I don't guarantee I'll respond, but I try to.
If you're so inclined, drop me a note.
I love to hear about how
Naive Cartographer is being used.
