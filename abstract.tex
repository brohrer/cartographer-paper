\label{sec:intro}

\textbf{tl;dr: A naive Bayes algorithm for learning Markov decision processes
with fuzzy state}

Fuzzy Naive Cartographer (FNC) is an online algorithm for
model-based reinforcement learning (RL), specificially it learns 
a Markov decision process and an action-value function.

FNC assumes a fuzzy state vector, namely
\begin{equation}
\mathcal{S} = [s_0, s_1, s_2, \cdots, s_i, \cdots, s_n] \mbox{ where } s_i \in [0, 1] 
\end{equation}
and makes the unusual implementation decision of
treating all elements of the state vector as conditionally independent, as in
Naive Bayes classification algorithms.

In FNC all actions are discrete.
\begin{equation}
\mathcal{A} = [a_0, a_1, a_2, \cdots, a_j, \cdots, a_m] \mbox{ where } a_j \in \{0, 1\} 
\end{equation}
FNC has two internal actions as well, a ``do-nothing" action as $a_{m+1}$ and
an ``average" action as $a_{m+2}$. The do-nothing action explicity represents 
refraining from taking an action, and the average action is always 1. It
allows FNC to learn typical feature-outcome patterns that may be independent
of agent actions.

Rewards are real valued, but FNC also allows them to be absent, and it
allows for multiple reward channels. 
\begin{equation}
\mathcal{R} = [r_0, r_1, r_2, \cdots, r_k, \cdots, r_l] \mbox{ where } r_k \in \{\mathbb{R}, \emptyset\}
\end{equation}

At each time step, $t$, an agent receives a state $s$ and reward $r$
and takes a set of actions $a$. The state from the following iteration is
represented as $s^{\prime}$ to show that it is the
outcome of previous state and actions.
FNC operates on a historical trace of prior states $\overline{s}$ where
\begin{equation}
\overline{s}_{i, t} = \max(s_{i, t}, \gamma_f \overline{s}_{i, t-1})
\end{equation}
In this recursive calculation $\gamma_f \in [0, 1]$ is a discount factor determining
how quickly the influence of prior states decays.

FNC also operates on a historical trace of state-action pairs,
$(\overline{\overline{s}, a})$, where
\begin{equation}
(\overline{\overline{s_i}, a_j})_t = \max(
(\overline{s_i}, a_j)_t , 
\gamma_r (\overline{\overline{s_i}, a_j})_{t-1})
\end{equation}
Again here $\gamma_r$ is a discount factor, and may be different than
$\gamma_f$. The activity of a state action pair is the product of the
individual activities of the feature and the action.
\begin{equation}
(\overline{s_i}, a_j) = \overline{s_i} a_j
\end{equation}

Transition probabilities are defined in terms of the all-time
sums of the transition
occurrences and the feature-action trace activities.
\begin{equation}
p(s^\prime_k | (\overline{\overline{s_i}, a_j}) = \frac
{\sum_t (\overline{\overline{s_i}, a_j}, s^\prime_k)_t}
{\sum_t (\overline{\overline{s_i}, a_j})_t + 1}
\end{equation}

Conditional transition probabilities at each time step are aggregated
over all active features with a weighted $\max()$ operation.
\begin{equation}
p(s^\prime_k | a_j) = \max_i (\overline{s_i} \mbox{ }
p(s^\prime_k | (\overline{\overline{s_i}, a_j}))
\end{equation}

FNC also generates an uncertainty estimate associated with each feature-action
pair.
\begin{equation}
\delta_{ij} = \frac{1}{\sum_t(\overline{\overline{s_i}, a_j}) + 1}
\end{equation}

The conditional uncertainty is aggregated in the same manner as transition
probabilities.
\begin{equation}
\delta_j = \max_i (\overline{s_i} \mbox{ } \delta_{ij})
\end{equation}

The estimate for reward $r_k$ and feature-action pair
$(\overline{\overline{s_i}, a_j})$
at time $t$ is $\hat{r}_{ijk, t}$ and is calculated incrementally via this
update rule
\begin{equation}
\hat{r}_{ijk, t} = \hat{r}_{ijk, t-1} + \alpha
(\overline{\overline{s_i}, a_j})_t
(r_{k,t} - \hat{r}_{ijk, t-1})
\end{equation}
where $\alpha \in [0, 1]$ is a learning rate.

The conditional reward for each action $a_j$
\begin{equation}
\hat{r}_{j} = \sum_k \left ( \frac{\sum_i \hat{r}_{ijk} \overline{s_i}}
{\sum_i \overline{s_i}} \right )
\end{equation}
